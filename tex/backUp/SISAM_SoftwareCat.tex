\documentclass[a4paper,10pt]{article}
%\documentclass[a4paper,10pt]{scrartcl}

\usepackage[utf8x]{inputenc}

\title{}
\author{}
\date{}

\pdfinfo{%
  /Title    ()
  /Author   ()
  /Creator  ()
  /Producer ()
  /Subject  ()
  /Keywords ()
}

\begin{document}
\maketitle

The Strategic Initiative on Stock Assessment Methods (SISAM) aims to advance knowledge on operation and development of stock assessments;
strengthen the stock assessment experts and the advice system; guide scientists to the most appropriate stock assessment software/methods
and generate ideas for the features of next generation assessment models.

SISAM will hold a World conference on stock assessment methods and publish a series of papers in the ICES Journal of Marine Science.
In addition an ICES Cooperative Research Report (CRR) review of state-of-the-art stock assessment methods will be published and an improved 
network of stock assessment developers created. Developing a repository of stock assessment methods is also under discussion.

The group thought this an important initiative and recommended that ICCAT collabarate with SISAM. Particularly, recognising the 
Commissions requirement [RES??] that the SCRS publish its scientific findings in the scientific peer-reviewed literature and the 
scientific work of the SCRS is based on the best available and peer-reviewed scientific deliverables, that the SCRS encourages that papers 
from ICCAT are presented at the World conference on stock assessment methods and published in IJMS.

In addition it is recommended that discussions are conducted with SISAM to ensure that the ICCAT software catalogue becomes part of a
worldwide repository of stock assessment methods. As part of these discussions the importance of evaluation and validation of software
must be considered. To ensure transpàrency all code must be published under a licence such as GPL. 

SISAM is also looking for test data sets to compare stock assessment models, North Atlantic Albacore would be a good case study for 
comparision since data are available by quarter, size, fleet and sex. There is also evidence of different substocks and potentially 
egg production data. In addition this is possible evidence of regime shift and environment drivers. Data also streaching back to 1940
and beyond, log book and other CPUE data are available,	 ...

The World Conference on Stock Assessment Methods (July 2013) included a workshop on testing assessment methods through simula-
tions. The exercise was made up of two steps applied to datasets from 14 representative fish stocks from around the world. Step 1 involved
applying stock assessments to datasets with varying degrees of effort dedicated to optimizing fit. Step 2 was applied to a subset of the stocks
and involved characteristics of given model fits being used to generate pseudo-data with error. These pseudo-data were then provided to
assessment modellers and fits to the pseudo-data provided consistency checks within (self-tests) and among (cross-tests) assessment
models. Although trends in biomass were often similar across models, the scaling of absolute biomass was not consistent across
models. Similar types of models tended to perform similarly (e.g. age based or production models). Self-testing and cross-testing
of models are a useful diagnostic approach, and suggested that estimates in the most recent years of time-series were the least robust.
Results from the simulation exercise provide a basis for guidance on future large-scale simulation experiments and demonstrate the
need for strategic investments in the evaluation and development of stock assessment methods.


%Victoria
here are my comments  about the text on Alb N Atlantic:
no data by sex, so far
Evidence of substocks, not  proven, just hypothesis so fa
no data on egg production, as a matter of fact very scarce data on matuirty and fecundity biology.
On the other hand, monitoring of catches and length information fairly adecuate in comparison to the other stocks standards.
Long time series of data as overall
Evidence of  regime sifts in relation to environment factors.
Cacthes are  not affected by management measures, since they had been at a level  below TAC, with one year exception.


%hairitz
 
"SISAM is looking for test data sets to compare stock assessment models,
North Atlantic Albacore would be a good case study for
comparision since data (catch, effort, size) are available by quarter, size and fleet since the 30’s.

There is also indications of potential subpopulations and possible evidence of regime
shift and environment drivers."








\end{document}
