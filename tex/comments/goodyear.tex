Laurie,

My antimalware/virus software had a fit with your message. It was quarantined and moved to junk. Even after I noticed and attempted to follow the links my email system seized. I did cut and paste the link to get to google forms, but got kicked out somehow and was informed that I had already completed filling it out and could not continue. Anyway:

 

1. FSIM. The current version of the population simulator FSIM is FSIM V4.0. I put the source code up on ResearchGate last year and continue to send out copies of the executables maybe a couple of time per year.  I am unsure how many folks are using it, but I do get occasional requests for support that by their nature indicate some sophistication on the part of the users. On reflection I think it may be useful to continue this software as a part of the Catalogue.

2. SEEPA. This simulator was long ago succeeded by LLSIM. I have received quite a few inquiries and I have distributed several copies of the code and executable. I don't know of anyone else who has actually applied the simulator for their purpose, but there may be somesuch. I do know of one instance where the code was translated to another language (Visual something ++ or another). I don't see much point in continuing to include SEEPA as part of the Catalog.

3. The version of the longline cpue simulator that I am about to provide to the methods WG is probably a good candidate for the catalogue. It will be a much simplified version of the LLSIM code, and thereby might be of more general utility (easier to use).

Phil 
